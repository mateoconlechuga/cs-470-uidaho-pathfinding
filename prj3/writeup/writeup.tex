\documentclass[11pt,letterpaper]{article}

% font
\usepackage{palatino}
\usepackage[T1]{fontenc}
\linespread{1.05}
\setlength{\parindent}{0em}
\setlength{\parskip}{1em}

% packages
\usepackage{graphicx}
\usepackage{placeins}
\usepackage{booktabs}
\usepackage{enumitem}
\usepackage{lastpage}
\usepackage{fancyhdr}
\usepackage{lipsum}
\usepackage{float}
\usepackage{subfig}
\usepackage{listings}
\usepackage{color}

\definecolor{codegreen}{rgb}{0,0.6,0}
\definecolor{codegray}{rgb}{0.5,0.5,0.5}
\definecolor{codepurple}{rgb}{0.58,0,0.82}
\definecolor{backcolour}{rgb}{255,255,255}
\definecolor{weborange}{RGB}{255,165,0}

\lstdefinestyle{mystyle}{
    backgroundcolor=\color{backcolour},
    commentstyle=\color{codegreen},
    keywordstyle=\color{blue},
    numberstyle=\tiny\color{codegray},
    stringstyle=\color{codepurple},
    emphstyle={\color{blue}},
    basicstyle=\footnotesize,
    breakatwhitespace=false,
    breaklines=true,
    captionpos=b,
    keepspaces=true,
    numbers=left,
    numbersep=5pt,
    showspaces=false,
    showstringspaces=false,
    showtabs=false,
    tabsize=2
    classoffset=1, % starting new class
    otherkeywords={self,<,.,;,-,!,=,~},
    morekeywords={self,<,.,;,-,!,=,~},
    keywordstyle=\color{or},
    classoffset=0,
}
\lstset{style=mystyle}

\usepackage[backend=bibtex8]{biblatex}
\usepackage[font=scriptsize,labelfont=bf]{caption}
\usepackage[left=.7in,right=.7in,top=1in,bottom=1in,headheight=13.6pt]{geometry}
\pagestyle{empty}

% enviornment
\graphicspath{{./figures/}}
\pagestyle{fancy}
\fancyhf[hl]{Project 3 - Genealogy Tree}
\fancyhf[hr]{2nd May 2018}
\author{Matthew Waltz}

% begin
\begin{document}

\begin{center}
   \textbf{CS 470: Project 3 - Genealogy Knowledge Base} \\
   \textbf{University of Idaho} \\
   \textbf{Matthew Waltz} \\
   \textbf{2nd May 2018} \\
\end{center}

\begin{abstract}
This project implements a Prolog knowledge base which can look up the gene relationships between characters in the hit TV series Game of Thrones as of Season 8.
With about 50 listed members of the tree, defined relationships can vary from queries such as 'son' and more relative terms such as 'ancestor' and 'descendant'.
The maximum depth of the tree is five generations.
The implementation works by taking a list of facts consisting of parents and genders for each individual in order to create the more complex rules and definitions.
\end{abstract}

\tableofcontents

\clearpage

\section{Summary}
The knowledge base of this project uses the two facts: parents and gender in order to build the list.
In prolog, a parent is defined as ``parent(a, b).'' with ``a'' being the parent of the child ``b''.

The two gender facts used are male and female, when written in prolog appear as ``male(a).'' and ``female(a).'' In this case, ``a'' relates to the individual that was previously defined in one of the parental relationships.

The above base facts are then used to create the more complex rules in the knowledge base. Furthermore, rules can use other rules in order to continue the levels of abstraction. For instance, a sibling would consist of two children with the same parent, and are not the same person. Using the ``sibling'' rule allows us to easily create the ``sister'' rule, which then further checks if the query is female.

The ``related'' rule is implemented simply if two people share the same ``ancestor'' query, which is defined below.

The ``ancestor'' query in the knowledge database uses an interesting recursive search method to iterate over all of the parental relationships in order to determine a match. Of course, the implementation of the ``descendants'' interface merely searches in the inverse direction, flipping the search tree in order to determine the descendants of an individual.


\section{Relationships}

The syntax of the rules for relationships follows a simple request in the form (<var,name>, <query name>), where ``var'' can be a capitalized variable in Prolog used for searching, or a ``name'' in order to return a boolean value check.
``query name'' relates to the search pattern.
This syntax is appended to the end of any of the names of the rule definition relationships.

Available relationships for use in the knowledge base is listed here:

parent, child, sibling, related, descendant, ancestor, mother, father, son, daughter, brother, sister, grandparent, grandfather, grandmother, grandchild, greatgrandparent, greatgrandchild, grandson, granddaughter, uncle, aunt, cousin, nephew, niece, secondcousin

\clearpage

\section{Results}
Query the father of Jon Snow:
\begin{verbatim}
| ?- father(A, snow).

A = rhaegar ? ;

no
\end{verbatim}

Query the second cousins of Stannis Baratheon:
\begin{verbatim}
| ?- secondcousin(A, stannis).

A = rhaegar ? ;

A = viserys ? ;

A = daenerys ? ;

(4 ms) no
\end{verbatim}


Query the ancestors of Robin of House Arryn:
\begin{verbatim}
| ?- ancestor(A, robin).

A = lysa ? ;

A = jon ? ;

A = hoster ? ;

A = minisa ? ;

A = robbyrt ? ;

no
\end{verbatim}

List all of the mothers in the knowledge base:
\begin{verbatim}
| ?- mother(A, B).

A = joanna
B = jaime ? ;

A = joanna
B = cersei ? ;

A = joanna
B = tyrion ? ;

A = cersei
B = joffrey ? ;

A = cersei
B = myrcella ? ;

A = cersei
B = tommen ? ;

A = rhaelle
B = steffon ? ;

A = lyanna
B = snow ? ;

A = minisa
B = lysa ? ;

A = minisa
B = edmure ? ;

A = minisa
B = catelyn ? ;

A = catelyn
B = robb ? ;

A = catelyn
B = sansa ? ;

A = catelyn
B = arya ? ;

A = catelyn
B = brandon ? ;

A = catelyn
B = rickon ? ;

A = lysa
B = robin ? ;

A = olenna
B = mace ? ;

no
\end{verbatim}

Check if Jon Snow and Daenerys Targarean are related (to keep the recursive rule simple, the output may be printed multiple times based on the depth of ancestors).
\begin{verbatim}
| ?- related(snow, daenerys).

true ? ; true ? ; true ? ; true ? ;

no
\end{verbatim}

Query if Stannis Baratheon has any uncles:
\begin{verbatim}
| ?- uncle(A, stannis).

no
\end{verbatim}

\section{Discussion}
The knowledge base performs quite well at gene-related (biological) relationships. However, it cannot represent non-biological relationships that well, either from a child outside of a marriage and/or adoption. It does quite well considering the incest of Game of Thrones though. Most queries can be answered with a simple or complex rule. Adding new rules is fairly simple, appending to the current list which can use the other available rules quickly makes it quite powerful. It would be nice to have a new rule and/or fact that is able to define adopted children in some way. I found that Prolog made this extremely easy, based on the logical processing that it entails. All in all, the knowledge base performs much better than I was initially thinking, and it makes sorting the huge tree a lot easier to understand.

One thing I found interesting is how easy it is to form queries. If I want everything in the knowledge base, I can just use two variables, i.e. parent(A, B). to list all of the parents. That was pretty neat.

\newgeometry{left=.6in,right=.6in,top=1in,bottom=.6in}
\section{Code Appendix}
\begin{lstlisting}[language=Prolog]
/* prolog family tree for game of thrones  */
/* matt waltz  */
/* spring 2018  */

/* house lannister */

parent(tywin, jaime).
parent(tywin, cersei).
parent(tywin, tyrion).

parent(joanna, jaime).
parent(joanna, cersei).
parent(joanna, tyrion).

parent(jaime, joffrey).
parent(jaime, myrcella).
parent(jaime, tommen).

parent(cersei, joffrey).
parent(cersei, myrcella).
parent(cersei, tommen).

/* house baratheon */

parent(ormund, steffon).
parent(rhaelle, steffon).

parent(steffon, robert).
parent(steffon, stannis).
parent(steffon, renly).

parent(stannis, shireen).

/* house targaryen */

parent(maekar, aemon).
parent(maekar, aegon).

parent(aegon, jaehaerys).
parent(aegon, rhaelle).

parent(jaehaerys, aerys).

parent(aerys, rhaegar).
parent(aerys, viserys).
parent(aerys, daenerys).

parent(rhaegar, snow).
parent(lyanna, snow).

/* house stark */

parent(rickard, brandon).
parent(rickard, eddard).
parent(rickard, benjen).
parent(rickard, lyanna).

parent(eddard, robb).
parent(eddard, sansa).
parent(eddard, arya).
parent(eddard, brandon).
parent(eddard, rickon).

/* house tully */

parent(robbyrt, hoster).
parent(robbyrt, brynden).

parent(hoster, lysa).
parent(hoster, edmure).
parent(hoster, catelyn).

parent(minisa, lysa).
parent(minisa, edmure).
parent(minisa, catelyn).

parent(catelyn, robb).
parent(catelyn, sansa).
parent(catelyn, arya).
parent(catelyn, brandon).
parent(catelyn, rickon).

parent(lysa, robin).

/* house tyrell */

parent(luthor, mace).

parent(olenna, mace).

parent(mace, margaery).
parent(mace, loras).

/* house martell */

parent(oberyn, elia).

parent(doran, elia).

/* house arryn */

parent(jon, robin).

/* male - female */

male(jon).
male(robin).
male(oberyn).
male(doran).
male(luthor).
male(mace).
male(loras).
male(robbyrt).
male(hoster).
male(bryden).
male(edmure).
male(rickard).
male(brandon).
male(eddard).
male(benjen).
male(robb).
male(brandon).
male(rickon).
male(ormund).
male(steffon).
male(robert).
male(stannis).
male(renly).
male(tywin).
male(jaime).
male(tyrion).
male(joffrey).
male(tommen).
male(maekar).
male(aemon).
male(aegon).
male(jaehaerys).
male(aerys).
male(rhaegar).
male(viserys).
male(snow).
female(rhaelle).
female(daenerys).
female(shireen).
female(myrcella).
female(cersei).
female(joanna).
female(sansa).
female(arya).
female(lyanna).
female(catelyn).
female(lysa).
female(minisa).
female(elia).
female(olenna).
female(margaery).

/* rule defines */

child(A, B) :-
    parent(B, A).

sibling(A, B) :-
    parent(C, A),
    parent(C, B),
    A \= B.

related(A, B) :-
    ancestor(C, A),
    ancestor(C, B).

descendant(A, B) :-
    ancestor(B, A).

ancestor(A, B) :-
    parent(A, B).

ancestor(A, B) :-
    parent(C, B),
    ancestor(A, C).

mother(A, B) :-
    parent(A, B),
    female(A).

father(A, B) :-
    parent(A, B),
    male(A).

son(A, B) :-
    child(A, B),
    male(A).

daughter(A, B) :-
    child(A, B),
    female(A).

sister(A, B) :-
    sibling(A, B),
    female(A),
    A \= B.

brother(A, B) :-
    sibling(A, B),
    male(A),
    A \= B.

grandparent(A, B) :-
    parent(A, C),
    parent(C, B).

grandfather(A, B) :-
    grandparent(A, B),
    male(A).

grandmother(A, B) :-
    grandparent(A, B),
    female(A).

grandchild(A, B) :-
    grandparent(B, A).

greatgrandparent(A, B) :-
    parent(P, B),
    grandparent(A, P).

greatgrandchild(A, B) :-
    greatgrandparent(B, A).

granddaughter(A, B) :-
    grandchild(A, B),
    female(A).

grandson(A, B) :-
    grandchild(A, B),
    male(A).

uncle(A, B) :-
    brother(A, C),
    child(B, C).

aunt(A, B) :-
    sister(A, C),
    child(B, C).

cousin(A, B) :-
    grandparent(C, A),
    grandparent(C, B),
    \+sibling(A, B),
    A \= B.

nephew(A, B) :-
    aunt(B, A),
    male(A);
    uncle(B, A),
    male(A).

niece(A, B) :-
    aunt(B, A),
    female(A);
    uncle(B, A),
    female(A).

secondcousin(A, B) :-
    greatgrandparent(C, A),
    greatgrandparent(C, B),
    \+sibling(A, B),
    \+cousin(A, B),
    A \= B.

\end{lstlisting}
\end{document}
